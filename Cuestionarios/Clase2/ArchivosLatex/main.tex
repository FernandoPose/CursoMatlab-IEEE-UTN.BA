\documentclass{bredelebeamer}

%%%%%%%%%%%%%%%%%%%%%%%%%%%%%%%%%%%%%%%%%%%%%%%%

\title[Programación en MatLAB]{Introducción a la programación con MatLAB}
\subtitle{Cuestionario "5x5"}

\author{Agustín - Andrés - Gabriel - Fernando\inst{1}}
\institute[UTN.BA]
{
  \inst{1}%
  Universidad Tecnológica Nacional\\
  Facultad Regional Buenos Aires
  }

\date{2018}

\subject{Taller de programación}

\logo{
\includegraphics[scale=0.15]{images/logo.png}
}

%%%%%%%%%%%%%%%%%%%%%%%%%%%%%%%%%%%%%%%%%%%%%%%%%%%%%%%%%%%%%%%%%%%%%
\begin{document}

\begin{frame}
  \titlepage 
\end{frame}

%%%%%%%%%%%%%%%%%%%%%%%%%%%%%%%%%%%%%%%%%%%%%%%%%%%%%%%%%%%%%%%%%%%%%

% Sección de operaciones vectoriales

%%%%%%%%%%%%%%%%%%%%%%%%%%%%%%%%%%%%%%%%%%%%%%%%%%%%%%%%%%%%%%%%%%%%%

\section{Cuestionario "5x5"}

\begin{frame}{Pregunta 1}
Siendo:
\begin{itemize}
\item $A = [9, 2, 3, 2, 4]$
\item $B = [1, 7, 8, 3, 2,7]$
\end{itemize}
El producto \textbf{A.*B} da por resultado un nuevo vector
\begin{itemize}
\item Verdadero
\item Falso
\end{itemize}
\end{frame}

\begin{frame}{Pregunta 1}
Siendo:
\begin{itemize}
\item $A = [9, 2, 3, 2, 4]$
\item $B = [1, 7, 8, 3, 2,7]$
\end{itemize}
El producto \textbf{A.*B} da por resultado un nuevo vector
\begin{itemize}
\item Verdadero
\item \textbf{Falso}
\end{itemize}
\end{frame}

\begin{frame}{Pregunta 2}
Siempre debe crearse una gráfica luego de agregarle el título y etiquetas. 
\begin{itemize}
\item Verdadero
\item Falso
\end{itemize}
\end{frame}

\begin{frame}{Pregunta 2}
Siempre debe crearse una gráfica luego de agregarle el título y etiquetas. 
\begin{itemize}
\item Verdadero
\item \textbf{Falso}
\end{itemize}
\end{frame}

\begin{frame}{Pregunta 3}
Las \textbf{variables locales} de una función pueden ser utilizadas por otras funciones y el script principal siempre que el programa no termine su ejecución.
\begin{itemize}
\item Verdadero
\item Falso
\end{itemize}
\end{frame}

\begin{frame}{Pregunta 3}
Las \textbf{variables locales} de una función pueden ser utilizadas por otras funciones y el script principal siempre que el programa no termine su ejecución.
\begin{itemize}
\item Verdadero
\item \textbf{Falso}
\end{itemize}
\end{frame}

\begin{frame}{Pregunta 4}
La función \textbf{disp()} permite controlar el número de decimales a mostrar por consola que contiene una variable numérica.
\begin{itemize}
\item Verdadero
\item Falso
\end{itemize}
\end{frame}

\begin{frame}{Pregunta 4}
La función \textbf{disp()} permite controlar el número de decimales a mostrar por consola que contiene una variable numérica.
\begin{itemize}
\item Verdadero
\item \textbf{Falso}
\end{itemize}
\end{frame}

\begin{frame}{Pregunta 5}
El siguiente script realiza un gráfico \textbf{y=f(x)} 
\lstinputlisting[xleftmargin=.1\textwidth]{scripts/ej1.m}
\begin{itemize}
\item Verdadero
\item Falso
\end{itemize}
\end{frame}

\begin{frame}{Pregunta 5}
El siguiente script realiza un gráfico \textbf{y=f(x)} 
\lstinputlisting[xleftmargin=.1\textwidth]{scripts/ej1.m}
\begin{itemize}
\item Verdadero
\item \textbf{Falso}
\end{itemize}
\end{frame}

\end{document}
