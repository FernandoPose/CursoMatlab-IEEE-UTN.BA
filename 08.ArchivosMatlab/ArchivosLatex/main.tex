\documentclass{bredelebeamer}
\usepackage{framed}
\usepackage{color}
\usepackage{wrapfig}\definecolor{shadecolor}{RGB}{255,0,0}
%%%%%%%%%%%%%%%%%%%%%%%%%%%%%%%%%%%%%%%%%%%%%%%%

\title[Programación en MatLAB]{Introducción a la programación con MatLAB}
\subtitle{Módulo 08 - Archivos en matlab}

\author{- AUTORES - \inst{1}}
\institute[UNIVERSIDAD]
{
  \inst{1}%
  - NOMBRE UNIVERSIDAD - 
  }

\date{AÑO}

\subject{Taller de programación}

\logo{
\includegraphics[scale=0.15]{images/logo.png}
}

%%%%%%%%%%%%%%%%%%%%%%%%%%%%%%%%%%%%%%%%%%%%%%%%%%%%%%%%%%%%%%%%%%%%%
\begin{document}

\begin{frame}
  \titlepage 
\end{frame}

%%%%%%%%%%%%%%%%%%%%%%%%%%%%%%%%%%%%%%%%%%%%%%%%%%%%%%%%%%%%%%%%%%%%%

% Archivos

%%%%%%%%%%%%%%%%%%%%%%%%%%%%%%%%%%%%%%%%%%%%%%%%%%%%%%%%%%%%%%%%%%%%%
\section{Archivos en Matlab}

\begin{frame}{Importación de datos}
Los datos se almacenan en muchos formatos diferentes. Algunos ejemplos podrían ser:
\begin{itemize}
\item Sonido: se almacena en un archivo .wav
\item Imagen: archivos .jpg
\item Tablas de excel: .xls
\end{itemize}
Para conocer los formatos admitidos por MATLAB escribir \textbf{doc fileformats} en la ventana de comandos. Aunque para nuevas opciones se recomienda recurrir a la web de \textbf{Mathworks}.
\end{frame}

\begin{frame}{Tipos de archivo soportados por Matlab}
\begin{table}[]
\centering
\begin{tabular}{|c|c|c|}
\hline
Tipo de archivo                                                                  & Extensión                                                         & Observación                                                                                                             \\ \hline
Texto                                                                            & \begin{tabular}[c]{@{}c@{}}.mat\\ .dat\\ .txt\end{tabular}        & \begin{tabular}[c]{@{}c@{}}Area de trabajo matlab\\ Datos ASCII\\ Datos ASCII\end{tabular}                              \\ \hline
\begin{tabular}[c]{@{}c@{}}Formatos comunes \\ de datos científicos\end{tabular} & \begin{tabular}[c]{@{}c@{}}.cdf\\ .fits\\ .hdf\end{tabular}       & \begin{tabular}[c]{@{}c@{}}Datos comunes\\ Transporte de imágenes\\ Datos jerárquicos\end{tabular}                      \\ \hline
Datos de hoja de cálculo                                                         & \begin{tabular}[c]{@{}c@{}}.xls\\ .wk1\\ .tiff\end{tabular}       & \begin{tabular}[c]{@{}c@{}}Hoja de cálculo Excel\\ Lotus 123\\ Archivo de imagen etiquetado\end{tabular}                \\ \hline
Datos de imágen                                                                  & \begin{tabular}[c]{@{}c@{}}.bmp\\ .jpeg o jpg\\ .gif\end{tabular} & \begin{tabular}[c]{@{}c@{}}Mapa de bits\\ Grupo experto fotográfico unido\\ Formato de intercambio gráfico\end{tabular} \\ \hline
Datos de audio                                                                   & \begin{tabular}[c]{@{}c@{}}.au\\ .wav\\.flac\end{tabular}                & \begin{tabular}[c]{@{}c@{}}Audio\\ Archivo wave Microsoft\\ Free Lossless Audio Codec\end{tabular}  
\\ \hline
Película                                                                         & .avi                                                              & Archivo intercalado audio/video                                                                                         \\ \hline
\end{tabular}
\end{table}
\end{frame}
%%%%%%%%%%%%%%%%%%%%%%%%%%%%%%%

\begin{frame}{Archivos disponibles}

\begin{block}{Utilidad}
Puedo conocer los archivos del directorio desde un script.
\end{block}

\lstinputlisting[xleftmargin=.3\textwidth]{scripts/dir.m}

\end{frame}

\begin{frame}{Archivos disponibles}

pero... ¿cómo accedo a esa información?
\begin{center}
\includegraphics[scale=0.3]{images/krusty}

\end{center}

\end{frame}

%%%%%%%%%%%%%%%%%%%%%%%%%%%%%%%%%%%%%
\begin{frame}{Importación de datos}
Conociendo el tipo de formato a importar puede utilizar una función de importación. Por ejemplo:

\begin{center}
\lstinputlisting[xleftmargin=.2\textwidth]{scripts/audioread.m}
\end{center}
Lee la canción \textit{ArrozConLeche}\\
\begin{block}{para esta función}
\textbf{audioread} soporta archivos:WAVE (.wav),
OGG (.ogg),
FLAC (.flac),
AU (.au),
AIFF (.aiff, .aif),
AIFC (.aifc) y en versiones recientes incluye MP3.
\end{block}
\end{frame}
%%%%%%%%%%%%%%%%%%%%%%%%%%%%%%%%%%%%%%%%%

\begin{frame}{Cargar datos desde un archivo ASCII}
\begin{itemize}
\item Un archivo ASCII contiene datos como texto
\item Todas las filas contienen el mismo número de datos.
\end{itemize}
Un ejemplo de archivo \textbf{text.txt} puede ser:
\begin{table}[]
\centering
\begin{tabular}{cccc}
\multicolumn{4}{c}{texto.txt} \\
2.5   & 7      & -3.2  & 4    \\
5     & 2.1    & 3.7   & 12   \\
-2    & -0.3   & 37    & -19  \\
4     & 3.2    & -1    & 0   
\end{tabular}
\end{table}
\begin{exampleblock}{Comando}
Ver comando: \textbf{load()}
\end{exampleblock}
Para cargar el archivo \textit{texto.txt} se escribe:\\
\begin{center}
\lstinputlisting[xleftmargin=.25\textwidth]{scripts/load.m}
\end{center}
\end{frame}

%%%%%%%%%%%%%%%%%%%%%%%%%%%%%%%%%%%%%%%%%%%
\begin{frame}{Función textread}
\begin{itemize}
\item Lee string y datos numéricos desde un archivo utilizando especificadores de conversión.
\item Los especificadores de conversión son por ejemplo formato de datos.
\item La función es útil cuando el archivo tiene un formato uniforme.
\end{itemize}
Un ejemplo de archivo \textbf{text.txt} puede ser:
\begin{table}[]
\centering
\begin{tabular}{cccc}
\multicolumn{4}{c}{texto.txt} \\
2.5   & 7      & -3.2  & 4    \\
5     & 2.1    & 3.7   & 12   \\
-2    & -0.3   & 37    & -19  \\
4     & 3.2    & -1    & 0   
\end{tabular}
\end{table}
\begin{exampleblock}{Comando}
\lstinputlisting[xleftmargin=.25\textwidth]{scripts/textread.m}
\end{exampleblock}

\end{frame}

\begin{frame}{archivo de texto}
\begin{exampleblock}{Comando}
escribir el siguiente archivo y leerlo:\\
Hola, soy un txt.
\end{exampleblock}
\begin{center}
\includegraphics[scale=0.6]{images/Fry}
\end{center}

\end{frame}
%%%%%%%%%%%%%%%%%%%%%%%%%%%%%%%%%%%%%%%%%%%
\begin{frame}{Función textread}
Para leer un archivo \textbf{.dat}. Por ejemplo:
\begin{table}[]
\centering
\begin{tabular}{cccc}
\multicolumn{4}{c}{personas.dat} \\
Manuel     & Hombre & 20 & Mayor \\
Camila     & Mujer  & 19 & Mayor \\
Juan       & Hombre & 33 & Mayor \\
Florencia  & Mujer  & 14 & Menor
\end{tabular}
\end{table}
\begin{block}{Formato}
\begin{center}
[A,B,C, ...] = textread(’archivo’,’formato’,N)\\
N es el número de filas que se deseen leer. El valor -1 permite leer todo el archivo.
\end{center}
\end{block}
Para cargar el archivo \textit{personas.dat} se escribe:\\
\begin{center}
\lstinputlisting[]{scripts/dat.m}
\end{center}
\end{frame}
%%%%%%%%%%%%%%%%%%%%%%%%%%%%%%%%%%%%%

\begin{frame}{Función dlmread}
\begin{exampleblock}{Comando}
Ver comando: \textbf{dlmread()}
\end{exampleblock}
La función \textbf{dlmread} permite leer una lista de valores desde un archivo separado por delimitadores.\\
Ej. Leer la siguiente tabla de datos separados por \textbf{;}
\begin{table}[]
\centering
\begin{tabular}{cccc}
\multicolumn{4}{c}{signal.dat} \\
4;     & 3;     & 2.4;  & 7    \\
-3;    & 0.33;  & 20;   & 12   \\
1;     & 1.7;   & 9;    & 12.4 \\
0.33;  & 9.3;   & -2;   & 3.3 
\end{tabular}
\end{table}
\begin{center}
\lstinputlisting[xleftmargin=.25\textwidth]{scripts/dlmread.m}
\end{center}
\end{frame}
%%%%%%%%%%%%%%%%%%%%%%%%%%%%%%%%%

\begin{frame}{Función xlsread}
\begin{exampleblock}{Comando}
Ver comando: \textbf{xlsread()}
\end{exampleblock}
\begin{itemize}
\item xlsread lee una hoja de cálculo de formato excel (xls)
\item Las celdas vacías o de texto serán retornadas como NaN en el dato
\end{itemize}
Ej. Leer la siguiente tabla de datos
\begin{table}[]
\centering
\begin{tabular}{cccc}
\multicolumn{4}{c}{datos.xls}                                                                                    \\ \hline
\multicolumn{1}{|c|}{4}    & \multicolumn{1}{c|}{3}    & \multicolumn{1}{c|}{2.4} & \multicolumn{1}{c|}{7}    \\ \hline
\multicolumn{1}{|c|}{-3}   & \multicolumn{1}{c|}{0.33} & \multicolumn{1}{c|}{20}  & \multicolumn{1}{c|}{12}   \\ \hline
\multicolumn{1}{|c|}{1}    & \multicolumn{1}{c|}{1.7}  & \multicolumn{1}{c|}{9}   & \multicolumn{1}{c|}{12.4} \\ \hline
\multicolumn{1}{|c|}{0.33} & \multicolumn{1}{c|}{9.3}  & \multicolumn{1}{c|}{-2}  & \multicolumn{1}{c|}{3.3}  \\ \hline
\end{tabular}
\end{table}
\begin{center}
\lstinputlisting[xleftmargin=.25\textwidth]{scripts/excel.m}
\end{center}
\end{frame}


\begin{frame}{Función xlsread}
Ej. Leer la siguiente tabla de datos
\begin{table}[]
\centering
\begin{tabular}{cccc}
\multicolumn{4}{c}{datos.xls}                                                                                              \\ \hline
\multicolumn{1}{|c|}{Canal 1} & \multicolumn{1}{c|}{Canal 2} & \multicolumn{1}{c|}{Canal 3} & \multicolumn{1}{c|}{Canal 4} \\ \hline
\multicolumn{1}{|c|}{4}      & \multicolumn{1}{c|}{3}      & \multicolumn{1}{c|}{2.4}    & \multicolumn{1}{c|}{7}       \\ \hline
\multicolumn{1}{|c|}{-3}     & \multicolumn{1}{c|}{0.33}   & \multicolumn{1}{c|}{20}     & \multicolumn{1}{c|}{12}      \\ \hline
\multicolumn{1}{|c|}{1}      & \multicolumn{1}{c|}{1.7}    & \multicolumn{1}{c|}{9}      & \multicolumn{1}{c|}{12.4}    \\ \hline
\multicolumn{1}{|c|}{0.33}   & \multicolumn{1}{c|}{9.3}    & \multicolumn{1}{c|}{-2}     & \multicolumn{1}{c|}{3.3}     \\ \hline
\end{tabular}
\end{table}
\begin{center}
\lstinputlisting[xleftmargin=.2\textwidth]{scripts/excel2.m}
\end{center}
\end{frame}

\begin{frame}{sobre los archivos xls}
\begin{alertblock}{Microsoft no es copado}
El uso de las funciones dedicadas depende de la aplicación \textbf{Microsoft Excel}
\end{alertblock}
\begin{center}
\includegraphics[scale=0.4]{images/bill}
\end{center}
\end{frame}
%%%%%%%%%%%%%%%%%%%%%%%%%%%%%%%%%%%%%%%%%

\begin{frame}{Exportación de datos}
Se verán tres formas de exportar datos:\\
\begin{itemize}
\item  save - Salva el espacio de trabajo (workspace)
\item  dlmwrite - Guarda un arreglo utilizando delimitadores
\item  xlswrite - Guarda un arreglo en una hoja de excel
\item  audiowrite - Crea un archivo de audio
\end{itemize}
\end{frame}
%%%%%%%%%%%%%%%%%%%%%%%%%%%%%%%%%%%%%%%

\begin{frame}{Función save}
\begin{exampleblock}{Comando}
Ver comando: \textbf{save} y \textbf{load}
\end{exampleblock}
La función \textbf{save} guarda el espacio de trabajo (workspace) en forma binaria creando un archivo .mat

La función \textbf{load} carga el archivo .mat recuperando el workspace salvado.
\begin{block}{Tener en cuenta}
\begin{center}
Para guardar el workspace con un determinado nombre se escribe:\\
\begin{center}
\lstinputlisting[xleftmargin=.3\textwidth]{scripts/save.m}
\end{center}
\end{center}
\end{block}
\end{frame}

\begin{frame}{Función save}
Ej. Ejecutar las siguientes líneas. Obtener conclusiones.
\begin{center}
\lstinputlisting[xleftmargin=.3\textwidth]{scripts/save2.m}
\end{center}

\end{frame}

\begin{frame}{Función dlmwrite}
\begin{exampleblock}{Comando}
Ver comando: \textbf{dlmwrite}
\end{exampleblock}
La función \textbf{dlmwrite} escribe el arreglo en un archivo delimitado por ASCII
\begin{exampleblock}{Comando}
Ver comando: \textbf{type}
\end{exampleblock}
La función \textbf{type} visualiza el archivo .txt
\end{frame}

\begin{frame}{Función dlmwrite}
Ej. Ejecutar las siguientes líneas. Obtener conclusiones.
\begin{center}
\lstinputlisting[xleftmargin=.25\textwidth]{scripts/dlmwrite.m}
\end{center}
\end{frame}

\begin{frame}{Función xlswrite}
\begin{exampleblock}{Comando}
Ver comando: \textbf{xlswrite}
\end{exampleblock}
La función \textbf{xlswrite} Guarda arreglo numérico o matriz en una hoja de Excel.

Ej. Ejecutar las siguientes líneas. Obtener conclusiones.
\begin{center}
\lstinputlisting[xleftmargin=.25\textwidth]{scripts/xlswrite.m}
\end{center}
\end{frame}
%%%%%%%%%%%%%%%%%%%%%%%%%%%%%%%%%%%
\begin{frame}{Función audioread}

\begin{exampleblock}{Comando}
Ver comando: \textbf{audiowrite}
\end{exampleblock}

Ej. Ejecutar las siguientes líneas. Obtener conclusiones.
\begin{center}
\lstinputlisting[xleftmargin=.25\textwidth]{scripts/audiowrite.m}
\end{center}

\end{frame}

%%%%%%%%%%%%%%%%%%%%%%%%%%%%%%%%%%%%%%%%%%%%%%%%%%%%%%%%%%%%%%%%%%%%%

% Sección de consultas

%%%%%%%%%%%%%%%%%%%%%%%%%%%%%%%%%%%%%%%%%%%%%%%%%%%%%%%%%%%%%%%%%%%%%

\section{Consultas}
\begin{frame}{Consultas}
\begin{center}
\includegraphics[scale=0.3]{images/consultas.png}
\end{center}
\end{frame}


%%%%%%%%%%%%%%%%%%%%%%%%%%%%%%%%%%%%%%%%%%%%%%%%%%%%%%%%%%%%%%%%%%%%%

% Sección de bibliografía

%%%%%%%%%%%%%%%%%%%%%%%%%%%%%%%%%%%%%%%%%%%%%%%%%%%%%%%%%%%%%%%%%%%%%

% \section{Bibliografia}

% \begin{frame}{Bibliografía}
% \begin{columns}
% \begin{column}{0.5\textwidth}
% \begin{center}
% \includegraphics[scale=0.4]{images/biblio1.png}
% \end{center}
% \end{column}
% \begin{column}{0.5\textwidth}
% \begin{center}
% \includegraphics[scale=0.5]{images/biblio2.png}
% \end{center}
% \end{column}
% \end{columns}
% \end{frame}

\end{document}
