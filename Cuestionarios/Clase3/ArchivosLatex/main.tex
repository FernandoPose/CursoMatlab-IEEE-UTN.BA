\documentclass{bredelebeamer}

%%%%%%%%%%%%%%%%%%%%%%%%%%%%%%%%%%%%%%%%%%%%%%%%

\title[Programación en MatLAB]{Introducción a la programación con MatLAB}
\subtitle{Cuestionario "5x5"}

\author{Agustín - Andrés - Gabriel - Fernando\inst{1}}
\institute[UTN.BA]
{
  \inst{1}%
  Universidad Tecnológica Nacional\\
  Facultad Regional Buenos Aires
  }

\date{2018}

\subject{Taller de programación}

\logo{
\includegraphics[scale=0.15]{images/logo.png}
}

%%%%%%%%%%%%%%%%%%%%%%%%%%%%%%%%%%%%%%%%%%%%%%%%%%%%%%%%%%%%%%%%%%%%%
\begin{document}

\begin{frame}
  \titlepage 
\end{frame}

%%%%%%%%%%%%%%%%%%%%%%%%%%%%%%%%%%%%%%%%%%%%%%%%%%%%%%%%%%%%%%%%%%%%%

% Sección de operaciones vectoriales

%%%%%%%%%%%%%%%%%%%%%%%%%%%%%%%%%%%%%%%%%%%%%%%%%%%%%%%%%%%%%%%%%%%%%

\section{Cuestionario "5x5"}

\begin{frame}{Pregunta 1}
La siguiente comparación resulta
\lstinputlisting[xleftmargin=.5\textwidth]{scripts/ej1.m}
\begin{itemize}
\item Verdadera
\item Falsa
\end{itemize}
\end{frame}

\begin{frame}{Pregunta 1}
La siguiente comparación resulta
\lstinputlisting[xleftmargin=.5\textwidth]{scripts/ej1.m}
\begin{itemize}
\item \textbf{Verdadera}
\item Falsa
\end{itemize}
\end{frame}

\begin{frame}{Pregunta 2}
La opción \textbf{otherwise} en la estructura de selección \textbf{switch/case} es obligatoria
\begin{itemize}
\item Verdadero
\item Falso
\end{itemize}
\end{frame}

\begin{frame}{Pregunta 2}
La opción \textbf{otherwise} en la estructura de selección \textbf{switch/case} es obligatoria
\begin{itemize}
\item Verdadero
\item \textbf{Falso}
\end{itemize}
\end{frame}

\begin{frame}{Pregunta 3}
La instrucción \textbf{continue} finaliza la ejecución del bucle mientras que la instrucción \textbf{break} pasa el control a la iteración siguiente.
\begin{itemize}
\item Verdadero
\item Falso
\end{itemize}
\end{frame}

\begin{frame}{Pregunta 3}
La instrucción \textbf{continue} finaliza la ejecución del bucle mientras que la instrucción \textbf{break} pasa el control a la iteración siguiente.
\begin{itemize}
\item Verdadero
\item \textbf{Falso}
\end{itemize}
\end{frame}

\begin{frame}{Pregunta 4}
La expresión \textbf{sum(A.*B)}  es equivalente a \textbf{dot(A,B)}
\begin{itemize}
\item Verdadero
\item Falso
\end{itemize}
\end{frame}

\begin{frame}{Pregunta 4}
La expresión \textbf{sum(A.*B)}  es equivalente a \textbf{dot(A,B)}
\begin{itemize}
\item \textbf{Verdadero}
\item Falso
\end{itemize}
\end{frame}

\begin{frame}{Pregunta 5}
La función \textbf{ones(2)} retorna:
\begin{itemize}
\item Una matriz cuadrada de unos
\item Un vector columna de unos
\item Un vector fila de unos
\item Matlab no permite ejecutar dicha función, la correcta es ones(2,2)
\end{itemize}
\end{frame}

\begin{frame}{Pregunta 5}
La función \textbf{ones(2)} retorna:
\begin{itemize}
\item \textbf{Una matriz cuadrada de unos}
\item Un vector columna de unos
\item Un vector fila de unos
\item Matlab no permite ejecutar dicha función, la correcta es ones(2,2)
\end{itemize}
\end{frame}

\end{document}
