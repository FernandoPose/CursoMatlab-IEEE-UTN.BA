\documentclass{bredelebeamer}

%%%%%%%%%%%%%%%%%%%%%%%%%%%%%%%%%%%%%%%%%%%%%%%%

\title[Programación en MatLAB]{Introducción a la programación con MatLAB}
\subtitle{Cuestionario "5x5"}

\author{Agustín - Andrés - Gabriel - Fernando\inst{1}}
\institute[UTN.BA]
{
  \inst{1}%
  Universidad Tecnológica Nacional\\
  Facultad Regional Buenos Aires
  }

\date{2018}

\subject{Taller de programación}

\logo{
\includegraphics[scale=0.15]{images/logo.png}
}

%%%%%%%%%%%%%%%%%%%%%%%%%%%%%%%%%%%%%%%%%%%%%%%%%%%%%%%%%%%%%%%%%%%%%
\begin{document}

\begin{frame}
  \titlepage 
\end{frame}

%%%%%%%%%%%%%%%%%%%%%%%%%%%%%%%%%%%%%%%%%%%%%%%%%%%%%%%%%%%%%%%%%%%%%

% Sección de operaciones vectoriales

%%%%%%%%%%%%%%%%%%%%%%%%%%%%%%%%%%%%%%%%%%%%%%%%%%%%%%%%%%%%%%%%%%%%%

\section{Cuestionario "5x5"}

\begin{frame}{Pregunta 1}
\textbf{x = sym(x)} y \textbf{syms x} son equivalentes
\begin{itemize}
\item Verdadero
\item Falso
\end{itemize}
\end{frame}

\begin{frame}{Pregunta 1}
\textbf{x = sym(x)} y \textbf{syms x} son equivalentes
\begin{itemize}
\item Verdadero
\item \textbf{Falso}
\end{itemize}
\end{frame}

\begin{frame}{Pregunta 2}
\textbf{x = sym('x' 'y')} y \textbf{syms x y} son equivalentes
\begin{itemize}
\item Verdadero
\item Falso
\end{itemize}
\end{frame}

\begin{frame}{Pregunta 2}
\textbf{x = sym('x' 'y')} y \textbf{syms x y} son equivalentes
\begin{itemize}
\item Verdadero
\item \textbf{Falso}
\end{itemize}
\end{frame}

\begin{frame}{Pregunta 3}
La función \textbf{solve()} permite calcular expresiones y ecuaciones
\begin{itemize}
\item Verdadero
\item Falso
\end{itemize}
\end{frame}

\begin{frame}{Pregunta 3}
La función \textbf{solve()} permite calcular expresiones y ecuaciones
\begin{itemize}
\item \textbf{Verdadero}
\item Falso
\end{itemize}
\end{frame}

\begin{frame}{Pregunta 4}
La función \textbf{ezplot()} requiere que el usuario determine el intervalo de graficación en caso contrario matlab arroja error.
\begin{itemize}
\item Verdadero
\item Falso
\end{itemize}
\end{frame}

\begin{frame}{Pregunta 4}
La función \textbf{ezplot()} requiere que el usuario determine el intervalo de graficación en caso contrario matlab arroja error.
\begin{itemize}
\item Verdadero
\item \textbf{Falso}
\end{itemize}
\end{frame}

\begin{frame}{Pregunta 5}
Siempre debe especificarse la variable independiente para utilizar la función \textbf{diff()} e \textbf{int()}
\begin{itemize}
\item Verdadero
\item Falso
\end{itemize}
\end{frame}

\begin{frame}{Pregunta 5}
Siempre debe especificarse la variable independiente para utilizar la función \textbf{diff()} e \textbf{int()}
\begin{itemize}
\item Verdadero
\item \textbf{Falso}
\end{itemize}
\end{frame}

\end{document}
