\documentclass{bredelebeamer}

%%%%%%%%%%%%%%%%%%%%%%%%%%%%%%%%%%%%%%%%%%%%%%%%

\title[Programación en MatLAB]{Introducción a la programación con MatLAB}
\subtitle{Cuestionario "5x5"}

\author{Agustín - Andrés - Gabriel - Fernando\inst{1}}
\institute[UTN.BA]
{
  \inst{1}%
  Universidad Tecnológica Nacional\\
  Facultad Regional Buenos Aires
  }

\date{2018}

\subject{Taller de programación}

\logo{
\includegraphics[scale=0.15]{images/logo.png}
}

%%%%%%%%%%%%%%%%%%%%%%%%%%%%%%%%%%%%%%%%%%%%%%%%%%%%%%%%%%%%%%%%%%%%%
\begin{document}

\begin{frame}
  \titlepage 
\end{frame}

%%%%%%%%%%%%%%%%%%%%%%%%%%%%%%%%%%%%%%%%%%%%%%%%%%%%%%%%%%%%%%%%%%%%%

% Sección de operaciones vectoriales

%%%%%%%%%%%%%%%%%%%%%%%%%%%%%%%%%%%%%%%%%%%%%%%%%%%%%%%%%%%%%%%%%%%%%

\section{Cuestionario "5x5"}

\begin{frame}{Pregunta 1}
El comando \textbf{clc} limpia el Workspace y el comando \textbf{clear} el Command Windows.
\begin{itemize}
\item Verdadero
\item Falso
\end{itemize}
\end{frame}

\begin{frame}{Pregunta 1}
El comando \textbf{clc} limpia el Workspace y el comando \textbf{clear} el Command Windows.
\begin{itemize}
\item Verdadero
\item \textbf{Falso}
\end{itemize}
\end{frame}

\begin{frame}{Pregunta 2}
\textbf{3pepe} es una variable y \textbf{'3pepe'} es una cadena de caracteres (string)
\begin{itemize}
\item Verdadero
\item Falso
\end{itemize}
\end{frame}

\begin{frame}{Pregunta 2}
\textbf{3pepe} es una variable y \textbf{'3pepe'} es una cadena de caracteres (string)
\begin{itemize}
\item Verdadero
\item \textbf{Falso}
\end{itemize}
\end{frame}

\begin{frame}{Pregunta 3}
\textbf{var = [4:2:12]} define el siguiente vector:
\begin{itemize}
\item $[4, 6, 8, 10, 12]$
\item $[4; 6; 8; 10; 12]$
\item $[4,5,6,7,8,9,10,11,12]$
\item Ninguna de las anteriores
\end{itemize}
\end{frame}

\begin{frame}{Pregunta 3}
\textbf{var = [4:2:12]} define el siguiente vector:
\begin{itemize}
\item $\textbf{[4, 6, 8, 10, 12]}$
\item $[4; 6; 8; 10; 12]$
\item $[4,5,6,7,8,9,10,11,12]$
\item Ninguna de las anteriores
\end{itemize}
\end{frame}

\begin{frame}{Pregunta 4}
\textbf{mean(v)}, siendo v una matriz de 3x3, calcula el valor medio de cada fila de la matriz v:
\begin{itemize}
\item Verdadero
\item Falso
\end{itemize}
\end{frame}

\begin{frame}{Pregunta 4}
\textbf{mean(v)}, siendo v una matriz de 3x3, calcula el valor medio de cada fila de la matriz v:
\begin{itemize}
\item Verdadero
\textbf{\item Falso}
\end{itemize}
\end{frame}

\begin{frame}{Pregunta 5}
La función \textbf{length(x)} retorna el tamaño del vector x ó la dimensión mas grande de la matriz x en cada caso.
\begin{itemize}
\item Verdadero
\item Falso
\end{itemize}
\end{frame}

\begin{frame}{Pregunta 5}
La función \textbf{length(x)} retorna el tamaño del vector x ó la dimensión mas grande de la matriz x en cada caso.
\begin{itemize}
\item \textbf{Verdadero}
\item Falso
\end{itemize}
\end{frame}

\end{document}
