\documentclass{bredelebeamer}

%%%%%%%%%%%%%%%%%%%%%%%%%%%%%%%%%%%%%%%%%%%%%%%%

\title[Programación en MatLAB]{Introducción a la programación con MatLAB}
\subtitle{Módulo 06 - Funciones definidas por el usuario}

\author{- AUTORES - \inst{1}}
\institute[UNIVERSIDAD]
{
  \inst{1}%
  - NOMBRE UNIVERSIDAD - 
  }

\date{AÑO}

\subject{Taller de programación}

\logo{
\includegraphics[scale=0.15]{images/logo.png}
}

%%%%%%%%%%%%%%%%%%%%%%%%%%%%%%%%%%%%%%%%%%%%%%%%%%%%%%%%%%%%%%%%%%%%%
\begin{document}

\begin{frame}
  \titlepage 
\end{frame}


%%%%%%%%%%%%%%%%%%%%%%%%%%%%%%%%%%%%%%%%%%%%%%%%%%%%%%%%%%%%%%%%%%%%%

% Funciones definidas por el usuario

%%%%%%%%%%%%%%%%%%%%%%%%%%%%%%%%%%%%%%%%%%%%%%%%%%%%%%%%%%%%%%%%%%%%%
\section{Funciones definidas por el usuario}

\begin{frame}{Introducción}
Hasta ahora:
\begin{center}
$cos(x)$
\end{center}
\begin{itemize}
\item Nombre de la función: \textbf{cos}
\item Argumento de entrada: \textbf{x}
\item Retorna un resultado\\
\end{itemize}
\begin{center}
\includegraphics[scale=0.5]{images/img43.png}
\end{center}
\end{frame}

\begin{frame}{Introducción}
Hasta ahora:
\begin{center}
$cos(x)$
\end{center}
\begin{itemize}
\item Nombre de la función: \textbf{cos}
\item Argumento de entrada: \textbf{x}
\item Retorna un resultado\\
\end{itemize}
\begin{block}{Tener en cuenta}
Las funciones definidas por el usuario funcionan del mismo modo.
\end{block}
\end{frame}

\begin{frame}{Funciones definidas por el usuario}
\begin{itemize}
\item Se crean en archivos .m
\item Comienzan con una línea de definición de función que contiene:
\begin{itemize}
\item la palabra reservada \textbf{function}
\item Una variable que defina la salida de función
\item Un nombre de función
\item Una variable que se use para el argumento de entrada
\end{itemize}
\end{itemize}
Sintaxis:
\lstinputlisting[xleftmargin=.2\textwidth]{scripts/ej1.m}
\end{frame}

\begin{frame}{Funciones definidas por el usuario}
\textbf{Consideraciones:}
\begin{itemize}
\item El nombre del archivo .m debe ser el mismo que el nombre de la función.
\item El nombre de la función debe comenzar con una letra.
\item El nombre de la función puede formarse con letras, números y guión bajo.
\item No se pueden usar nombres reservados.
\textbf{\item Permite cualquier longitud.}
\end{itemize}
\end{frame}

\begin{frame}{Ejercicio práctico 8}
\begin{center}
Realice una función que convierte minutos en segundos.
\end{center}
\begin{alertblock}{Importante}
Matlab puede acceder a funciones definidas por el usuario únicamente si están almacenadas en el directorio de trabajo actual.
\end{alertblock}
\end{frame}

\begin{frame}{Funciones con entradas y salidas múltiples}
Funciones de múltiples entradas y salidas. Sintaxis:\\
\lstinputlisting[xleftmargin=.03\textwidth]{scripts/ej2.m}
\begin{center}
\includegraphics[scale=0.15]{images/img40.png}
\end{center}
\end{frame}

\begin{frame}{Funciones con entradas y salidas múltiples}
Siendo la forma de invocar a la función:
\lstinputlisting[xleftmargin=.05\textwidth]{scripts/ej3.m}
\end{frame}

\begin{frame}{Ejercicio práctico 9}
\begin{enumerate}
\item Escribir una función para multiplicar dos vectores punto a punto.
\item Escribir una función que dado un valor de tiempo calcule la distancia, velocidad y aceleración de un automóvil teniendo en cuenta:
\begin{itemize}
\item aceleracion = 0.5*t
\item velocidad = aceleracion*t
\item posición = vel *t
\end{itemize}
\end{enumerate}
\end{frame}

\begin{frame}{Funciones sin entrada o salida}
Funciones sin entradas y salidas. Sintaxis:\\
\lstinputlisting[xleftmargin=.3\textwidth]{scripts/ej4.m}
Siendo la forma de invocar a la función:
\lstinputlisting[xleftmargin=.35\textwidth]{scripts/ej5.m}
\end{frame}

\begin{frame}{Variables locales y globales}
\begin{itemize}
\item \textbf{Variables locales:} Son las variables definidas dentro de una función. Sólo existen para el uso de la función.
\item \textbf{Variables globales:} 
\end{itemize}
\end{frame}

\begin{frame}{Variables locales y globales}
\begin{center}
\includegraphics[scale=0.4]{images/img41.png}
\end{center}
\end{frame}

\end{document}
